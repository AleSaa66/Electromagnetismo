 \begin{appendix}
 \renewcommand{\theHchapter}{A\arabic{chapter}}
\renewcommand{\theHsection}{A\arabic{section}}
\renewcommand{\theHsubsection}{A\arabic{subsection}}

\chapter{Integrales útiles en electromagnetismo} \label{Integrales-Utiles}

Sea $a > 0$, es habitual encontrarse con las siguientes integrales en la resolución de problemas del electromagnetismo.
\begin{align}
    I_1 &= \int \frac{1}{(\sqrt{x^2+a^2})^3} \,dx = \frac{x}{a^2 \sqrt{x^2+a^2}} + C \label{A-I1}\\
    I_2 &= \int \frac{1}{x^2+a^2} \,dx = \frac{1}{a} \arctan{\left( \frac{x}{a}\right)} + C  \label{A-I2}\\
    I_3 &= \int \frac{x}{(\sqrt{x^2+a^2})^3} \,dx = - \frac{1}{\sqrt{x^2+a^2}} + C  \label{A-I3}\\
    I_4 &= \int \frac{x}{\sqrt{x^2+a^2}} \,dx = \sqrt{x^2+a^2} + C \label{A-I4} \\
    I_5 &= \int \frac{1}{\sqrt{x^2+a^2} } \,dx = \ln(x+\sqrt{x^2+a^2}) + C \label{A-I5}
\end{align}

\chapter{Movimiento de una carga en un campo magnético uniforme} \label{Mov-B-cte}

\end{appendix}

